\documentclass[10pt,oneside,slovak,a4paper]{article}

\usepackage[slovak]{babel}
%\usepackage[T1]{fontenc}
\usepackage[IL2]{fontenc} % lepšia sadzba písmena Ľ než v T1
\usepackage[utf8]{inputenc}
\usepackage{graphicx}
\usepackage{url} % príkaz \url na formátovanie URL
\usepackage{hyperref} % odkazy v texte budú aktívne (pri niektorých triedach dokumentov spôsobuje posun textu)
\usepackage{booktabs}

\usepackage{cite}
%\usepackage{times}

\pagestyle{headings}

\title{Kahoot! v E-learningu\thanks{Semestrálny projekt v predmete Metódy inžinierskej práce, ak. rok 2020/21, vedenie: Fedor Lehocki}} % meno a priezvisko vyučujúceho na cvičeniach

\author{Adam Škurla\\[2pt]
	{\small Slovenská technická univerzita v Bratislave}\\
	{\small Fakulta informatiky a informačných technológií}\\
	{\small \texttt{xskurla@stuba.sk}}
	}

\date{\small 15. október 2020}



\begin{document}

\maketitle

\begin{abstract}
V tejto práci sledujeme použitie Kahoot! v E-learningu. Kahoot! je vzdelávacia platforma založená na hre a interaktívnej forme kvízov, kde študenti môžu ihneď reagovať a zlepšuje to ich komunikáciu s učiteľom počas učiva. Kahoot! je aplikácia, ktorá sa momentálne používa najmä kvôli tomu, že študent dostane takmer okamžitú spätnú väzbu. V tejto práci , by sme sa chceli zaoberať s tým, ako študenti a aj učitelia zlepšili svoje výsledky práve pomocou aplikácie Kahoot! a to najmä v učení cudzích jazykov. Študentom sa zvýšila úspešnosť na predmete práve vďaka tejto aplikácii. Dôvodom je najmä to, že väčšina študentov si túto aplikáciu pochvaľovala najmä preto lebo mohli lepšie a rýchlejšie reagovať a učitelia mohli vytvárať omnoho lepšie otázky a dať tak študentom omnoho lepšie skúsenosti s cudzím jazykom v písanej forme.
\end{abstract}



\section{Úvod}

V posledných rokoch môžeme sledovať zvýšený nárast v interaktívnych technológiách. Tento nárast umožnilo najmä to, že dnes vlastní smartfón alebo iné technologické zariadenie až 3,5 miliard ľudí, čo je 44,81\% svetovej populácie a toto číslo bude už len narastať. Technológie sa už dostali vo veľkom aj do škôl, kde uľahčujú prácu učiteľom a zároveň zlepšujú motiváciu žiakov sa učiť. Najmä interaktívne technológie a aplikácie si získali na škole veľké publikum. Žiaci totižto viac obľubujú hravú formu učenia, vo forme súťaží, kvízov alebo niečoho podobného. Prebúdza to v nich väčšiu pozornosť a skôr sa tak sústredia a zapamätajú si nové učivo. Pre učiteľov je zase pozitívum to, že väčšinou tieto technológie a aplikácie nie sú ťažké a tak sa s nimi naučia rýchlo pracovať, zároveň tak dostávajú lepšiu spätnú väzbu od svojich žiakov ako rozumejú učivu, lebo im súčasne môže odpovedať aj viac žiakov naraz. Medzi také obľúbené interaktívne aplikácie patrí Socrative, Quizlet alebo Kahoot! Práve Kahoot je z týchto aplikácii najslávnejší a najčastejšie používaní. 


\section{Nejaká časť} \label{nejaka}

Z obr.~\ref{f:rozhod} je všetko jasné. 

\begin{figure*}[tbh]
\centering
%\includegraphics[scale=1.0]{diagram.pdf}
Aj text môže byť prezentovaný ako obrázok. Stane sa z neho označný plávajúci objekt. Po vytvorení diagramu zrušte znak \texttt{\%} pred príkazom \verb|\includegraphics| označte tento riadok ako komentár (tiež pomocou znaku \texttt{\%}).
\caption{Rozhodujúci argument.}
\label{f:rozhod}
\end{figure*}


<<<<<<< HEAD

=======
>>>>>>> 298a0562592e45ba4da56c05b112c56b65884fd4
\section{Iná časť} \label{ina}

Základným problémom je teda\ldots{} Najprv sa pozrieme na nejaké vysvetlenie (časť~\ref{ina:nejake}), a potom na ešte nejaké (časť~\ref{ina:nejake}).\footnote{Niekedy môžete potrebovať aj poznámku pod čiarou.}

Môže sa zdať, že problém vlastne nejestvuje\cite{Coplien:MPD}, ale bolo dokázané, že to tak nie je~\cite{Czarnecki:Staged, Czarnecki:Progress}. Napriek tomu, aj dnes na webe narazíme na všelijaké pochybné názory\cite{PLP-Framework}. Dôležité veci možno \emph{zdôrazniť kurzívou}.


\subsection{Nejaké vysvetlenie} \label{ina:nejake}

Niekedy treba uviesť zoznam:

\begin{itemize}
\item jedna vec
\item druhá vec
	\begin{itemize}
	\item x
	\item y
	\end{itemize}
\end{itemize}

Ten istý zoznam, len číslovaný:

\begin{enumerate}
\item jedna vec
\item druhá vec
	\begin{enumerate}
	\item x
	\item y
	\end{enumerate}
\end{enumerate}


\subsection{Ešte nejaké vysvetlenie} \label{ina:este}

\paragraph{Veľmi dôležitá poznámka.}
Niekedy je potrebné nadpisom označiť odsek. Text pokračuje hneď za nadpisom.



\section{Dôležitá časť} \label{dolezita}




\section{Ešte dôležitejšia časť} \label{dolezitejsia}




\section{Záver} \label{zaver} % prípadne iný variant názvu



%\acknowledgement{Ak niekomu chcete poďakovať\ldots}


% týmto sa generuje zoznam literatúry z obsahu súboru literatura.bib podľa toho, na čo sa v článku odkazujete
\bibliography{literatura}
\bibliographystyle{plain} % prípadne alpha, abbrv alebo hociktorý iný
<<<<<<< HEAD
\end{document}
=======
\end{document}
>>>>>>> 298a0562592e45ba4da56c05b112c56b65884fd4
